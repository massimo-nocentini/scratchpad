
\chapter{Number theory}

\section{Multiply}

In order to multiply two integers $x$ and $y$ we can derive a procedure
by simple manipulation:
\begin{displaymath}
    x\,y = 2x\,\frac{y}{2} = 2\left(x\,\frac{y}{2}\right)
\end{displaymath}
this allows us to build a recursive definition. To accomplish that, observe
that $2x$ is always an integer; on the contrary, $\frac{y}{2}$ is an integer if $y$ is even.
Therefore we have to split the definition by cases on the parity of $y$:
\begin{align}
    x\,y &= \varsubsdevice{2(x\,q)}{y=2q}\\
    x\,y &= \varsubsdevice{2\left(x\,\frac{2q+1}{2}\right)}{y=2q+1}            
            = \varsubsdevice{2\left(x\,q\right)+x}{y=2q+1}
\end{align}
adding base cases for the recursion, as many as remainder classes of $\equiv_{2}$, and 
abstracting the previous definition we get:
\begin{align}
    x\,0 &= 0\\
    x\,1 &= x\\
    x\,y &= \varsubsdevice{2\left(x\,q\right)+x\,r}{y=2q+r}
\end{align}
rewriting variable substitution device $\varsubsdevice{\ldots}{\ldots}$ with pattern matching as in functional
programming languages we get:
\begin{align}
    x\,0 &= 0\\
    x\,1 &= x\\
    x\,(2\,q+r) &= 2\left(x\,q\right)+x\,r
\end{align}
Using previous derivation, we can manually unfold a generic product:
\begin{align}
    x\,y &= \varsubsdevice{2\left(x\,q_{0}\right)+x\,r_{0}}{y=2q_{0}+r_{0}}\\
         &= \varsubsdevice{2\left(2\left(x\,q_{1}\right)+x\,r_{1}\right)+x\,r_{0}}{q_{0}=2q_{1}+r_{1}}\\
         &= \varsubsdevice{2\left(2\left(2\left(x\,q_{2}\right)+x\,r_{2}\right)+x\,r_{1}\right)+x\,r_{0}}{q_{1}=2q_{2}+r_{2}}
\end{align}
this process eventually terminates exactly when $q_{\floor{\log_{2}{y}}}=0$. Therefore:
\begin{align}
    x\,y &= x\,\sum_{k=0}^{\floor{\log_{2}{y}}}{r_{k}2^{k}}
\end{align}
and, provided that $x\neq0$, we get back to the binary representation of $y$:
\begin{align}
    y &= \sum_{k=0}^{\floor{\log_{2}{y}}}{r_{k}2^{k}}
\end{align}


\section{Exponentiation}

Using previous section about multiplication as a track to follow, 
we do the same steps to derive exponentiation of two integers $x$ and $y$:
\begin{displaymath}
    x^{y} = x^{\left(\frac{y}{2}\right)^{2}} = \left(x^{\left(\frac{y}{2}\right)}\right)^{2} 
\end{displaymath}
again proceed by cases respect the parity of $y$:
\begin{displaymath}
    \begin{split}
        x^{y} &= \varsubsdevice{\left(x^{q}\right)^{2}}{y=2q}\\
        x^{y} &= \varsubsdevice{\left(x^{\frac{2q+1}{2}}\right)^{2}}{y=2q+1}
               = \varsubsdevice{\left(x^{q}\right)^{2}\,x}{y=2q+1}\\
    \end{split}
\end{displaymath}
introducing bases for the recursion and abstracting into one line we have:
\begin{displaymath}
    \begin{split}
        x^{0} &= 1\\
        x^{1} &= x\\
        x^{y} &= \varsubsdevice{\left(x^{q}\right)^{2}\,x^{r}}{y=2q+r}\\
    \end{split}
\end{displaymath}
Similarly, do manual unfolding:
\begin{displaymath}
    \begin{split}
        x^{y} &= \varsubsdevice{\left(x^{q_{0}}\right)^{2}\,x^{r_{0}}}{y=2q_{0}+r_{0}}\\
              &= \varsubsdevice{\left(\left(x^{q_{1}}\right)^{2}\,x^{r_{1}}\right)^{2}\,x^{r_{0}}}{q_{0}=2q_{1}+r_{1}}\\
              &= \varsubsdevice{\left(\left(\left(x^{q_{2}}\right)^{2}\,x^{r_{2}}\right)^{2}\,x^{r_{1}} \right)^{2}\,x^{r_{0}}}
                    {q_{1}=2q_{2}+r_{2}}\\
              &= \varsubsdevice{\left(x^{q_{2}}\right)^{2^{3}}\,x^{2^{2}r_{2}}\,x^{2r_{1}}\,x^{r_{0}}}{q_{1}=2q_{2}+r_{2}}\\
    \end{split}
\end{displaymath}
as before, the unfolding process stops exactly when $q_{\floor{\log_{2}{y}}}=0$, therefore:
\begin{align}
    x^{y} &= x^{\sum_{k=0}^{\floor{\log_{2}{y}}}{r_{k}2^{k}}}
\end{align}
Previous derivation can be used as follows:
\begin{displaymath}
    3^{29} = 3^{2^{4}}3^{2^{3}}3^{2^{2}}3^{2^{0}}
\end{displaymath}
since $29 = {11101}_{2}$.
